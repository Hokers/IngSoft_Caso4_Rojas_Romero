\section{Anexos}
\subsection{Glosario}

\begin{enumerate}
	\item \textbf{APP:} Acrónimo de aplicación.
	\item \textbf{Codificación:} Acción de generar código de programación.
	\item \textbf{DET:} Data Element Types, Son la cantidad de datos relacionados a un elemento de la aplicación.
	\item \textbf{EI:} Entrada externa, son los datos entregados a la aplicación.
	\item \textbf{EO:} Salida externa, son los datos entregados por la aplicación.
	\item \textbf{EQ:} Consulta externa, hace referencia a las consultas que realice la aplicación a otros sistemas.
	\item \textbf{EIF:} Ficheros de interfaces externas, grupo de datos relacionados lógicamente, se mantienen fuera de la aplicación.
	\item \textbf{FTR:} File Type Referenced, son la cantidad de conecciones a los diferentes grupos de datos.
	\item \textbf{FP:} Function points, es el valor de medida entregado por las funcionalidades de una aplicación, están ayudan a definir la complejidad de un proyecto.
	\item \textbf{HH:} Horas hombre, es una unidad de medida para medir el esfuerzo de un trabajo según las horas de trabajo por persona.
	\item \textbf{ILF:} Ficheros lógicos internos, grupo de datos relacionados lógicamente, se mantienen dentro de la aplicación.
	\item \textbf{LOC:} Lines of code, es una unidad de medida que denota el valor de un código por su cantidad de líneas.
	\item \textbf{Pieza, PH  o Pieza Histórica:} Es el nombre usado para referirse a los elementos del juego que representan a los propios elementos de los museos que están en exposición.
	\item \textbf{RRSS:} Acrónimo de redes sociales.
	\item \textbf{UI:} User Interface, es el tipo de vista que se ocupa en una aplicación que permite al usuario interactuar con la aplicación.
\end{enumerate}















