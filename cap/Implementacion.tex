\section{Implementación}

\subsection{Código fuente completo (parcial)}
El código de la app se hizo en base al diagrama UML presentado anteriormente, este se encuentra en un repositorio de github el cual se encuentra público para la revisión completa de este.

El código contiene tanto bibliotecas externas, como los scripts creados por el equipo de programación. 

\href{https://github.com/pokelocos/Prototipo_Ing_Software/tree/master/repo_ingSoftware}{Repositorio del Prototipo de `Museo en Casa'. (Hacer Click en el texto para ir al link)}



\subsection{Dependencias}

\begin{itemize}

\item Vuforia: Se utilizo esta librería para implementar las funcionalidades de Realidad Aumentada dentro del motor gráfico Unity.

\item Android SDK: Se utilizo esta SDK para realizar los builds para los dispositivos de Android.

\item Unity Engine package: Se utilizo esta librería para todas las funcionalidades que eran necesarias para la app, ya sea interacción con la UI, elementos del juego, etc.

\item Native Share for Android & iOS: Se utilizo esta librería para poder compartir las piezas de museos que hayan sido descubiertas por los usuarios de la app.


\end{itemize}