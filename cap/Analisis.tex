\section{Análisis}

\subsection{Contexto}

\subsubsection{Descripción General}

Este proyecto busca, mostrar los museos de manera más interactiva utilizando RA y código QR, permitiendo a sus usuarios no solo acceder a las piezas expuestas en los museos, sino que también visualizar el interior de los museos y poder interactuar con dichas piezas que historicas.


\subsubsection{Descripción de Clientes y Usuarios:}

Esta aplicación apunta dos tipos de usuarios principalmente, el primero 1 más relevante dentro del objetivo entregar la cultura es a estudiantes de finales de 2do ciclo (enseñanza básica) y estudiantes de 3er ciclo (enseñanza media) es decir de 6to básico a 4to medio, este grupo esta compuestos por personas de alrededor de 11 a 19 años de edad y que están en un periodo de formación educacional, además consideramos como segundo grupo clave de usuarios a personas que ya estén interesadas en museos, cualquiera sea su edad, y por sobretodo a las interesadas en las tecnologías realidad aumentada.

En base al grupo principal de usuarios se destaca que los mayores interesados en educarlos son sus padres o tutores además de la institución educativa a la cual pertenecen y su docentes, es por esto que como principal generador de demanda estarán estas instituciones, y como principal cliente se acudirá al ministerio de educación y entes preocupados por la educación de los jóvenes chilenos.

Como alianza extra se abrirá suscripción a museos a que estén interesados en aparecer dentro de esta aplicación, esto les dará una nueva canal de comunicación con las personas interesadas en ellos, además de la obtención de información por medio de nuestra aplicación.


\subsection{Especificación de Requerimientos}

\subsubsection{Funciones del Sistema}

\begin{longtable}{|c|p{10cm}|c|}
\hline 
Ref\# & Función & Categoría (E/O/S) \\ 
\hline 
1.1 & Iniciar juego al presionar en el medio de la pantalla con la app abierta. & E \\ 
\hline 
1.2 & Mostrar título del juego & E \\ 
\hline 
1.3 & Escanear código & E \\ 
\hline 
1.4 & Mostrar museo correspondiente al código Escaneado & E \\ 
\hline 
2.1 & Visualizar pieza en 3d situada en el museo & E \\ 
\hline 
2.2 & Obtener modelo de las piezas & O \\ 
\hline 
2.3 & Interactuar con la pieza & E \\ 
\hline 
2.4 & Visualizar pieza en 3d en el panel de información & E \\ 
\hline 
2.5 & Desplegar visualizador de pieza & E \\ 
\hline 
2.6 & Desplegar ventana de manipulación de pieza & E \\ 
\hline 
2.7 & Obtener información de las piezas & O \\ 
\hline 
2.8 & Mostrar información de las piezas & E \\ 
\hline 
3.1 & Activar/Desactivar menú desplegable de opciones y características & E \\ 
\hline 
3.2 & Silenciar aplicación & S \\ 
\hline 
3.3 & Redireccionar a pagina de la aplicación & S \\ 
\hline 
3.4 & Mostrar Guia/Tutorial de uso básico de la app & E \\ 
\hline 
3.5 & Tutorial de manejo de pieza & E \\ 
\hline 
4.1 & Construir mensaje al compartir en RRSS & O \\ 
\hline 
4.2 & Obtener información de la pieza para compartir en RRSS & O \\ 
\hline 
4.3 & 
Desplegar menú para compartir en RRSS
 & E \\ 
\hline 
4.4 & Obtener información del museo para compartir en RRSS & O \\ 
\hline 
5.1 & Desplegar menú de museos & E \\ 
\hline 
5.2 & Mostrar museos visitados y no visitados & E \\ 
\hline 
5.3 & Obtener información de museos & O \\ 
\hline 
5.4 & Desplazarse entre los museos & E \\ 
\hline 
5.5 & Cerrar ventana de museos & E \\ 
\hline 
5.6 & Descargar QR del museo & O \\ 
\hline 
5.7 & Buscador de museo & E \\ 
\hline 
5.8 & Zoom Museo & E \\ 
\hline 
5.9 & Desplazarse por el museo virtual & E \\ 
\hline 
5.1.1 & Mostrar la descarga del QR del museo & E \\ 
\hline 
5.1.2 & Mostrar información de los museos & E \\ 
\hline 
6.1 & Desplegar menu de piezas & E \\ 
\hline 
6.2 & Retroceder al menú de museos & E \\ 
\hline 
6.3 & Desplazarse entre las piezas & E \\ 
\hline 
6.4 & Mostrar piezas obtenidas y no descubiertas & E \\ 
\hline 
6.5 & Obtener información de piezas & O \\ 
\hline 
6.6 & Cerrar ventana de piezas & E \\ 
\hline 
6.7 & Zoom Pieza & E \\ 
\hline 
6.8 & Rotación pieza & E \\ 
\hline 
7.1 & Desplegar menu para visualizar logros & E \\ 
\hline 
7.2 & Cerrar ventana de logros & E \\ 
\hline 
7.3 & Obtener información de logros & O \\ 
\hline 
7.4 & Mostrar logros obtenidos y no completados & E \\ 
\hline 
7.5 & Desplazarse entre los logros & E \\ 
\hline 
7.6 & Visualizar nuevo logro & E \\ 
\hline 
\end{longtable} 

\subsubsection{Atributos del Sistema}

\begin{longtable}{|c|p{3.5cm}|p{10cm}|}
\hline 
Ref\# & Atributo & Detalle y limitación \\ 
\hline 
AT1.1 & Tiempo de respuesta del escaneo & Menor a 1.5 segundos para informar al usuario que se detectó el código y menor a 7 para mostrar el contenido final.  \\ 
\hline 
AT1.2 & Uso en móvil gama media & XXXX  \\ 
\hline
AT1.3 & Interfaz implementada con iconos & Cualquier usuario debe ser capaz de orientarse en la aplicación sin importar su edad (dentro de las estipuladas en los grupos usuarios) o idioma (que saber español o inglés no sea un requerimiento para ocupar la app).  \\ 
\hline
AT1.4 & Información presentada de manera simple y legible & Tanto la información de las piezas como la de los museos debe ser presentada en un formato que no abrume al usuario, debe ser digerible y fácil de leer.  \\ 
\hline
AT1.5 & Tutoriales simples y autoexplicativo & Cada nuevo sistema dentro de la aplicación debe ser enseñado a través de un tutorial que muestre una imagen de ejemplo de lo que se está explicando y un texto que describa la situación como mínimo.	\\ 
\hline
AT1.6 & Mantener al usuario interesado, evitar que se frustre & Las piezas a encontrar  siempre deben tener un leve feedback para que el jugador note su presencia, además si el usuario pasa más de 40 segundos en una habitación con una pieza esta debe mostrarse de manera aún más notable, si esto supera los 90 segundos se debe mostrar de forma evidente su ubicación.  \\ 
\hline
AT1.7 & XXXX & XXXX  \\ 
\hline
\end{longtable}

\begin{longtable}{|c|p{4.7cm}|c|p{4.7cm}|c|}
\hline 
Ref\# & Función & Categoría (E/O/S) & Atributo & Categoría (E/R/D)\\ 
\hline 
RX.X & Escaneo codigo & E & Tiempo de respuesta del escaneo & R \\ 
\hline 
RX.X & Visualizar pieza 3D & E & Uso en móvil gama media & E \\ 
\hline
RX.X & Obtener información de las piezas & E & Información presentada de manera simple y legible & R \\ 
\hline
RX.X & Mostrar guia/tutorial de uso básico de app & E & Tutoriales simples y autoexplicativo & D \\ 
\hline
\end{longtable}

\subsubsection{Atributos por Función}

\newpage
\subsection{Actores}

\subsection{Casos de Uso}
\subsubsection{Caso de Uso Esencial}
\subsubsection{Diagrama de Caso de Uso}
\subsubsection{Contrato}
\subsubsection{Modelo Conceptual}
\subsubsection{Diagrama de Secuencia o Colaboración}
\subsubsection{Priorización}

\subsection{Modelo de Dominio}
\subsubsection{Entidades Reconocidas}
\subsubsection{Modelo de Dominio}
\subsubsection{Matriz de Rastreabilidad}