\section{Introducción}
El presente documento es la guía de desarrollo e implementación del proyecto ``Museo en casa'' con la finalidad de tener una nueva vía para acercar la cultura a los hogares chilenos.

\subsection{Propósito}
Este proyecto busca conectar por una nueva vía a las población chilena con museos locales, acercando el conocimiento de los museos a su alrededor hasta sus propias manos, del museo a su teléfono móvil.

Se busca que esta app llegue a todo público, aunque está enfocada en amantes de la cultura y aficionados por la realidad aumentada, buscando entregarle una nueva forma de descubrir la historia y la cultura.

En el presente documento se describe el proceso de desarrollo para la aplicación ``Museo en casa'', ademas se analizaran y evaluaran los requerimientos y funciones necesarias para poder llevar esto acabo.

\subsection{Descripción breve del problema}
El COVID-19 a traído consigo grandes cambios con respecto a cómo es nuestro estilo de vida, y a generando para la mayoría de la población mundial un gran cambio en su estilo de vida, por esto la población ha tenido que adaptarse y usar aun mas de lo que ya se usaban la tecnología con el fin de seguir en contacto con el resto del mundo,  junto con esta adaptación también debemos destacar que cosas como salir de la casa para ir al colegio o a un paseo familiar se transformaron en actividades que no están permitidas lo que a ocasionado que actividades como ir al museo local sea algo impensado. Podemos notar en los datos del gobierno de chile la gran baja en la asistencia a los museos desde fines de 2019 en chile\footnote{Ver en: https://www.museoschile.gob.cl/sitio/Contenido/Institucional/90496:Estadisticas-generales}, este fenómeno ha alejado a los museos de las personas que podrían estar interesadas.

Debido a este fenómeno es que se plantea la opción de llevar la experiencia de ir a un museo al hogar de todas las familias chilena de una nueva forma digital aprovechando las tecnologías de realidad aumentada, la aplicación ``Museo en casa'' cubrirá la necesidad de descubrimiento de la población chilena y la posibilidad de llevar la cultura chilena a la comodidad del hogar.  
