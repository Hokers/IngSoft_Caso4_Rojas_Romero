\section{Introducción}
\color{red}
La vida actual en la que vivimos está rodeada de aparatos tecnológicos y diversos elementos que han extendido la forma en que las personas interactúan entre ellas y con su entorno, dentro de estas interacciones es sencillo notar que existe un incremento en en interés de las personas en visitar lugares culturales como pueden centros históricos, lugares patrimoniales y museos (datos obtenidos de : SNM subdirección Nacional de Museos).

2019 - 1.210.075\\
2018 - 1.390.952 - 18.73 m\\
2017 - 1.252.418 - 18.47 m\\
2016 - 1.276.068 - 18.21 m\\
2015 - 1.112.809 - 17.97 m\\
2014 - 714.137 - 17.76 m\\
2013 - 719.675 - 17.57 m\\
2012 - 685.103 - 17.40 m\\
2011 - 632.765 - 17.23 m\\
2010 - 561.461 - 17.06 m\\
2009 - 682.144 - 16.89 m\\
2008 - 667.679 - 16.71 m\\
https://www.museoschile.gob.cl/sitio/Contenido/Institucional/90496:Estadisticas-generales\\
\color{black}

\subsection{Propósito}
\color{red}
Este proyecto está dirigido a todo público, enfocado en amantes de la cultura y aficionados por la realidad aumentada.
El propósito de esta aplicación es conectar por esta nueva vía a las personas con los museos locales, acercandolos hasta sus propias manos.
 en el siguiente documento se documenta el diseño de la aplicación en conjunto a su plan de desarrollo desde la concepción de la idea hasta la implementación y mantenimiento de esta.

\color{black}

\subsection{Descripción breve del problema}
\color{red}
(posible problema) el COVID-19 a traído consigo grandes cambios con respecto a cómo es nuestro estilo de vida, y generando para la mayoría de la población mundial el tener que adaptarse y usar aun mas de lo que ya se usaban la tecnología para poder conectarse con el resto de personas,  junto con esta adaptación también debemos destacar que cosas como salir de la casa para ir al colegio o salir de la casa a en un paseo familiar se transformaron en actividades que no están permitidas. (datos del gob de chile ->) podemos notar con los datos del gob la gran baja en la asistencia a los museos desde marzo del 2020 en chile (insertar cifra), este fenómeno ha alejado a los museos de las personas que podrían estar interesadas.

\color{black}