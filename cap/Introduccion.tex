\section{Introducción}
\color{red}
La vida actual en la que vivimos está rodeada de aparatos tecnológicos y diversos elementos que han extendido la forma en que las personas interactúan entre ellas y con su entorno, dentro de estas interacciones es sencillo notar que existe un incremento en en interés de las personas en visitar lugares culturales como pueden centros históricos, lugares patrimoniales y museos (datos obtenidos de : SNM subdirección Nacional de Museos).

2019 - 1.210.075\\
2018 - 1.390.952 - 18.73 m\\
2017 - 1.252.418 - 18.47 m\\
2016 - 1.276.068 - 18.21 m\\
2015 - 1.112.809 - 17.97 m\\
2014 - 714.137 - 17.76 m\\
2013 - 719.675 - 17.57 m\\
2012 - 685.103 - 17.40 m\\
2011 - 632.765 - 17.23 m\\
2010 - 561.461 - 17.06 m\\
2009 - 682.144 - 16.89 m\\
2008 - 667.679 - 16.71 m\\
https://www.museoschile.gob.cl/sitio/Contenido/Institucional/90496:Estadisticas-generales\\
\color{black}

\subsection{Propósito}
\color{red}
El propocito de esta aplicacion es conectar a las personas por esta nueva via de informacion acercando los museos locales hasta sus propias manos.
\color{black}

\subsection{Descripción breve del problema}
\color{red}
(posible problema) el COVID-19 a traido concigo grandes cambios con respecto a como es nuestro estilo de vida, y generando para la mayoria de la poblacion mundial el tener que adapatarse y usar aun mas de lo que ya se usaban la tecnologia para poder conectarse con el respo de personas,  junto con esta adaptacion tambien debemos destacar que cosas como salir de la casa para ir al coleguio o salir de la casa a en un paseo familiar se transformaron en actividades que no estan permitidas. (datos del gob de chile ->) podemos notar con los datos del gob la gran baja en la asistencia a los museos desde marzo del 2020 en chile (incertar cifra), este fenomeno a alejado a los museos de las personas que podrian estar interesadas.
\color{black}