\section{Planificación del Trabajo}

\subsection{Descripción del grupo de trabajo}
El grupo para este proyecto esta conformado por:

\begin{tabular}{|c|c|c|}
\hline 
Nombre & Alias & Capacidades de profesión \\ 
\hline 
José Rojas & JR & Programador Junior, Artista, Diseñador de Videojuegos \\ 
\hline 
Nicolás Romero & NR & Programador Junior, Artista, Diseñador de Videojuegos \\ 
\hline 
\end{tabular} 

Descripción del general del trabajo de JR: Se encargara de la gestión del proyecto, el
plan del proyecto, trabajara en conjunto con su compañero para hacer los requerimientos,
realizara tanto el diseño inicial como detallado de la app, hará la mitad de la codificación y por ultimo hará la creación de la UI.

Descripción del general del trabajo de NR: Se encargara de parte de los requerimientos,
trabajara en conjunto con su compañero para hacer el plan del proyecto,ademas realizara la
prueba unitaria, prueba funcional, prueba de integración, prueba de la aplicación finalizada
y por ultimo, hara la mitad de la codificación del proyecto y la creación de Assets 3D.


\subsection{Estimación de esfuerzo}
\subsubsection{Complejidad de Archivos}
La siguiente tabla muestra las diferentes funciones necesaria para el correcto funcionamiento del software.

\begin{tabular}{|p{5cm}|p{8cm}|}
\hline 
Tipos de parámetros & Nombres \\ 
\hline 
EI (Entradas Externas) & Obtener Pieza Historica (PH), Buscar PH \\ 
\hline 
EO (Salidas Externas) & Compartir Modelo 3d o Foto de PH en RRSS.
Lista de piezas historicas obtenidas, Biblioteca de trofeos obtenidos.
 \\ 
\hline 
EQ (Consultas Externas) & Visualizar PH, Rotar PH, Ver información de PH, Hacer Zoom in a PH, Hacer Zoom out a
PH.
 \\ 
\hline 
ILF (Ficheros lógicos internos) & Datos de PH, Datos de Trofeos. \\ 
\hline 
EIF (Ficheros de interfaces externas) & Obtener piezas historicas compartidas en
RRSS.
 \\ 
\hline 
\end{tabular} 

\begin{tabular}{|p{5cm}|p{8cm}|}
\hline 
Objetos & Datos necesarios \\ 
\hline 
Pieza Historica & Autor,Nombre de la pieza, Año de creación, Material
usado, dimensiones, Ubicación en el museo, Nombre del
Museo. Nombre del fotografo de la pieza. Detalles de la
pieza y datos interesantes. \\ 
\hline 
Trofeo & Tipo de Trofeo, Nombre de trofeo, detalles de trofeo. \\ 
\hline 
Museo & Nombre, ubicación, fecha de construcción, lista de piezas
historicas. \\ 
\hline 
\end{tabular} 

\subsubsection{Complejidad de las funcionalidades, cálculo de FP, grados de influencia, y otros cálculos}
En las siguientes tabla se muestra la complejidad de las diferentes funcionalidades ligadas al software.

\begin{tabular}{|c|c|c|c|}
\hline 
Entradas Externas EI & • & • & • \\ 
\hline 
Nombre & DET & FTR & Complex \\ 
\hline 
Obtener Pieza Historica & 4 & 1 & Low \\ 
\hline 
Buscar Pieza Historica & 2 & 1 & Low \\ 
\hline 
Obtener Trofeo & 3 & 1 & Low \\ 
\hline 
Escanear Codigo QR del Muse & 3 & 1 & Low \\ 
\hline 
\end{tabular}

\begin{tabular}{|c|c|c|c|}
 \hline 
 Salidas Externas EO & • & • & • \\ 
 \hline 
 Nombre & DET & FTR & Complex \\ 
 \hline 
 Compartir Modelo 3d o Foto de PH en RRSS & 3 & 2 & Low \\ 
 \hline 
 Biblioteca de piezas historicas obtenidas & 3 & 1 & Low \\ 
 \hline 
 Biblioteca de trofeos obtenidos. & 3 & 1 & Low \\ 
 \hline 
 \end{tabular} 
 
 \begin{tabular}{|c|c|c|c|}
  \hline 
  Consultas Externas EQ & • & • & • \\ 
  \hline 
  Nombre & DET & FTR & Complex \\ 
  \hline 
  Visualizar Pieza Historica & 2 & 1 & Low \\ 
  \hline 
  Rotar Pieza Historica & 2 & 1 & Low \\ 
  \hline 
  Ver info. de Pieza Historica & 2 & 1 & Low \\ 
  \hline 
  Zoom in a Pieza Historica & 2 & 1 & Low \\ 
  \hline 
  Zoom out a Pieza Historica & 2 & 1 & Low \\ 
  \hline 
  Ver info. de Museo & 2 & 1 & Low \\ 
  \hline 
  Visualizar Museo & 2 & 1 & Low \\ 
  \hline 
  Ver info. de Trofe & 2 & 1 & Low \\ 
  \hline 
  \end{tabular} 
  
  \begin{tabular}{|c|c|c|c|}
   \hline 
   Ficheros lo´gicos internos ILF & • & • & • \\ 
   \hline 
   Nombre & DET & FTR & Complex \\ 
   \hline 
   Datos de Piezas Historicas & 11 & 1 & Low \\ 
   \hline 
   Datos de Trofeos & 4 & 1 & Low \\ 
   \hline 
   Datos de Museos & 4 & 2 & Low \\ 
   \hline 
   \end{tabular}  
   
   \begin{tabular}{|c|c|c|c|}
   \hline 
   EIF (Ficheros de interfaces externas) & • & • & • \\ 
   \hline 
   Nombre & DET & FTR & Complex \\ 
   \hline 
   Obtener piezas historicas compartidas en RRSS. & 3 & 2 & Low \\ 
   \hline 
   \end{tabular} 
   
   Ahora podemos calcular los puntos de función sin ajustar, TUFP.
   
   \begin{tabular}{|p{3cm}|p{1.5cm}|p{1.9cm}|p{1.5cm}|p{1.9cm}|p{1.5cm}|p{1.9cm}|}
   \hline 
   • & \multicolumn{2}{c|}{Bajo} & \multicolumn{2}{c|}{Medio} & \multicolumn{2}{c|}{Alto} \\ 
   \hline 
   • & Número & Puntuación & Número & Puntuación & Número & Puntuación \\ 
   \hline 
   Ficheros Lógicos Internos a la Aplicación & 3 & 7 & 0 & 10 & 0 & 15 \\ 
   \hline 
   Ficheros Lógicos Externos a la Aplicación
    & 1 & 5 & 0 & 7 & 0 & 10 \\ 
   \hline 
   Input Externos & 4 & 3 & 0 & 4 & 0 & 6 \\ 
   \hline 
   Outputs Externos
    & 3 & 4 & 0 & 5 & 0 & 7 \\ 
   \hline 
   Queries & 7 & 3 & 0 & 4 & 0 & 6 \\ 
   \hline 
   • & • & 71 & • & 0 & • & 0 \\ 
   \hline 
   \end{tabular} 
   
   El total de TUFP = 53 , cálculo y tabla generada a partir de la planilla de cálculo.
   
   Grados de influencia y otros cálculos
   
   Calculamos los grados de influencia a partir de la planilla de calculo y podemos obtener lo siguiente:
   
   \begin{tabular}{|c|c|}
   \hline 
   TOTAL GRADOS DE INFLUENCIA & 11 \\ 
   \hline 
   Puntos de Función Ajustados & 53 \\ 
   \hline 
   VAF & 0,76 \\ 
   \hline 
   \end{tabular} 
   
\subsubsection{C.T Jones / COCOMO II}

\begin{tabular}{|c|c|}
\hline 
FP & 53 \\ 
\hline 
Loc en POO  & 20 \\ 
\hline 
\end{tabular} 

\begin{tabular}{|p{8cm}|c|}
\hline 
C.T Jones & • \\ 
\hline 
Métrica  & Cálculo \\ 
\hline 
Estimación de meses de desarrollo & 4,894522709 \\ 
\hline 
Número de personas necesarias para el desarrollo (NP) & 0,3533333333 \\ 
\hline 
Estimación del esfuerzo en personas/mes  & 1,729398024 \\ 
\hline 
Estimación de esfuerzo en horas hombre (160 horas x mes) 160 horas, ya que
, se considera 1 jornada diaria de 8 horas & 276,7036838 \\ 
\hline 
\end{tabular} 

Estimación de esfuerzo en base a C.T. Jones 1.996 Software Estimating Rules of Thumb” y B. Boehm COCOMO II en fase de Early Design

\begin{tabular}{|c|c|}
\hline 
COCOMO II & • \\ 
\hline 
Métrica  & Cálculo \\ 
\hline 
Conversión de FP a LOC. & 1060 \\ 
\hline 
\end{tabular}

Calculo de esfuerzo con COCOMO II con entrada de 1260 LOC y con factores de escala nominales.

\begin{tabular}{|p{3cm}|c|c|c|}
 \hline 
 Estimación & Optimista & Conservador & Pesimista \\ 
 \hline 
 Meses de desarrollo & 4,6 & 5,2 & 5,9 \\ 
 \hline 
 Personas necesarias & 0,4 & 0,6 & 0,8 \\ 
 \hline 
 Personas/mes & 2 & 3 & 4,5 \\ 
 \hline 
 Esfuerzo en horas hombre (160 horas x mes) & 294,4 & 499,2 & 755,2 \\ 
 \hline 
 \end{tabular}
 
\begin{tabular}{|c|c|}
  \hline 
  LOC Equivalentes & 1060 \\ 
  \hline 
\end{tabular}

Estimación del costo monetario en base al esfuerzo en HH conservador. Definiremos 2 perfiles:

Programador Junior: Perfil requerido para realizar la codificación crítica en base al diseño del software, gestión del proyecto, requerimientos, el plan del proyecto y los diseños base de la aplicación.

Artista Junior: Perfil requerido para la creación de assets 3d y UI para los diseños creados para el proyecto.

Estimación del coste de producir el software en 5.2 meses con 499 HH, bajo el modelo conservador de COCOMO II 

\begin{tabular}{|p{4cm}|p{2.2cm}|p{2.4cm}|p{2.7cm}|p{2cm}|}
\hline 
Costo del producto  & • & • & • & • \\ 
\hline 
Etapa    & Distribución en \% & Distribución de T (HH) & Costo x Recurso & Recurso \\ 
\hline 
Gestión del proyecto & 3 & 14,97 & Programador Junior & \$44.910 \\ 
\hline 
Requerimientos & 8 & 39,92 & Programador Junior & \$119.760 \\ 
\hline 
Plan del proyecto & 1 & 4,99 & Programador Junior & \$14.970 \\ 
\hline 
Diseño inicial & 8 & 39,92 & Programador Junior & \$119.760 \\ 
\hline 
Diseño detallado & 8 & 39,92 & Programador Junior & \$119.760 \\ 
\hline 
Codificación & 48 & 239,52 & Programador Junior & \$718.560 \\ 
\hline 
Documentación de usuario & 1 & 4,99 & Programador Junior & \$14.970 \\ 
\hline 
Prueba unitaria & 3 & 14,97 & Programador Junior & \$44.910 \\ 
\hline 
Prueba funcional & 3 & 14,97 & Programador Junior & \$44.910 \\ 
\hline 
Prueba de integración & 4 & 19,96 & Programador Junior & \$59.880 \\ 
\hline 
Prueba de la app & 6 & 29,94 & Programador Junior & \$89.820 \\ 
\hline 
Creación de Assets 3D & 4 & 19,96 & Artista Junior & \$39.920 \\ 
\hline 
Creación de UI & 3 & 14,97 & Artista Junior & \$29.940 \\ 
\hline 
• & 100 & 499 & • & \$1.392.210 \\ 
\hline 
\end{tabular} 



   
  





\subsection{Asignación de recursos}

\subsection{Planificación temporal de actividades}