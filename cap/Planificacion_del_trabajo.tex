\section{Planificación del Trabajo}
En la presente sección se describirán los integrantes de equipo, sus capacidades, el plan de desarrollo de la aplicación y la estimación de trabajo dedicada a esta.

\subsection{Descripción del grupo de trabajo}
El grupo para este proyecto esta conformado por:

\begin{longtable}{|c|c|c|}
\hline 
Nombre & Alias & Capacidades de profesión \\ 
\hline 
José Rojas & JR & Programador Junior, Artista, Diseñador de Videojuegos \\ 
\hline 
Nicolás Romero & NR & Programador Junior, Artista, Diseñador de Videojuegos \\ 
\hline 
\caption{Tabla de Descripción del grupo de trabajo}
\label{tab1}
\end{longtable} 

Descripción del general del trabajo de JR: Se encargará de la gestión del proyecto, el
plan del proyecto, trabajara en conjunto con su compañero para hacer los requerimientos,
realizara tanto el diseño inicial como detallado de la app, hará la mitad de la codificación y por ultimo hará la creación de la UI.

Descripción del general del trabajo de NR: Se encargará de parte de los requerimientos,
trabajara en conjunto con su compañero para hacer el plan del proyecto,ademas realizara la
prueba unitaria, prueba funcional, prueba de integración, prueba de la aplicación finalizada
y por ultimo, hara la mitad de la codificación del proyecto y la creación de Assets 3D.


\subsection{Estimación de esfuerzo}
En la siguiente sección se especifica la estimación de esfuerzo para la realización de esta app tomando en cuenta los las diferentes entradas y complejidades de archivo además de sus costes de producción.

\subsubsection{Complejidad de Archivos}
La siguiente tabla muestra las diferentes funciones necesaria para el correcto funcionamiento del software.

\begin{longtable}{|p{5cm}|p{8cm}|}
\hline 
Tipos de parámetros & Nombres \\ 
\hline 
EI (Entradas Externas) & Obtener Pieza Historica (PH), Buscar PH \\ 
\hline 
EO (Salidas Externas) & Compartir Modelo 3d o Foto de PH en RRSS.
Lista de piezas historicas obtenidas, Biblioteca de trofeos obtenidos.
 \\ 
\hline 
EQ (Consultas Externas) & Visualizar PH, Rotar PH, Ver información de PH, Hacer Zoom in a PH, Hacer Zoom out a
PH.
 \\ 
\hline 
ILF (Ficheros lógicos internos) & Datos de PH, Datos de Trofeos. \\ 
\hline 
EIF (Ficheros de interfaces externas) & Obtener piezas historicas compartidas en
RRSS.
 \\ 
\hline 
\caption{Tabla de Complejidad de Archivos}
\label{tab2}
\end{longtable} 

\begin{longtable}{|p{5cm}|p{8cm}|}
\hline 
Objetos & Datos necesarios \\ 
\hline 
Pieza Historica & Autor,Nombre de la pieza, Año de creación, Material
usado, dimensiones, Ubicación en el museo, Nombre del
Museo. Nombre del fotografo de la pieza. Detalles de la
pieza y datos interesantes. \\ 
\hline 
Trofeo & Tipo de Trofeo, Nombre de trofeo, detalles de trofeo. \\ 
\hline 
Museo & Nombre, ubicación, fecha de construcción, lista de piezas
historicas. \\ 
\hline 
\caption{Tabla de Objetos y Datos necesarios}
\label{tab3}
\end{longtable} 

\subsubsection{Complejidad de las funcionalidades, cálculo de FP, grados de influencia, y otros cálculos}
En las siguientes tabla se muestra la complejidad de las diferentes funcionalidades ligadas al software.

\begin{longtable}{|c|c|c|c|}
\hline 
Entradas Externas EI &   &   &   \\ 
\hline 
Nombre & DET & FTR & Complex \\ 
\hline 
Obtener Pieza Historica & 4 & 1 & Low \\ 
\hline 
Buscar Pieza Historica & 2 & 1 & Low \\ 
\hline 
Obtener Trofeo & 3 & 1 & Low \\ 
\hline 
Escanear Codigo QR del Muse & 3 & 1 & Low \\ 
\hline 
\caption{Tabla de Entradas Externas}
\label{tab4}
\end{longtable}

\begin{longtable}{|c|c|c|c|}
 \hline 
 Salidas Externas EO &   &   &   \\ 
 \hline 
 Nombre & DET & FTR & Complex \\ 
 \hline 
 Compartir Modelo 3d o Foto de PH en RRSS & 3 & 2 & Low \\ 
 \hline 
 Biblioteca de piezas historicas obtenidas & 3 & 1 & Low \\ 
 \hline 
 Biblioteca de trofeos obtenidos. & 3 & 1 & Low \\ 
 \hline 
 \caption{Tabla de Salidas Externas}
\label{tab5}
 \end{longtable}
 
 \begin{longtable}{|c|c|c|c|}
  \hline 
  Consultas Externas EQ &  &  &  \\ 
  \hline 
  Nombre & DET & FTR & Complex \\ 
  \hline 
  Visualizar Pieza Historica & 2 & 1 & Low \\ 
  \hline 
  Rotar Pieza Historica & 2 & 1 & Low \\ 
  \hline 
  Ver info. de Pieza Historica & 2 & 1 & Low \\ 
  \hline 
  Zoom in a Pieza Historica & 2 & 1 & Low \\ 
  \hline 
  Zoom out a Pieza Historica & 2 & 1 & Low \\ 
  \hline 
  Ver info. de Museo & 2 & 1 & Low \\ 
  \hline 
  Visualizar Museo & 2 & 1 & Low \\ 
  \hline 
  Ver info. de Trofe & 2 & 1 & Low \\ 
  \hline 
  \caption{Tabla de Consultas Externas}
  \label{tab6}
  \end{longtable} 
  
  \begin{longtable}{|c|c|c|c|}
   \hline 
   Ficheros lógicos internos ILF &  &  &  \\ 
   \hline 
   Nombre & DET & FTR & Complex \\ 
   \hline 
   Datos de Piezas Historicas & 11 & 1 & Low \\ 
   \hline 
   Datos de Trofeos & 4 & 1 & Low \\ 
   \hline 
   Datos de Museos & 4 & 2 & Low \\ 
   \hline
   \caption{Tabla de Ficheros lógicos internos}
   \label{tab7} 
   \end{longtable}  
   
   \begin{longtable}{|c|c|c|c|}
   \hline 
   EIF (Ficheros de interfaces externas) &  &  &  \\ 
   \hline 
   Nombre & DET & FTR & Complex \\ 
   \hline 
   Obtener piezas historicas compartidas en RRSS. & 3 & 2 & Low \\ 
   \hline 
   \caption{Tabla de Ficheros de interfaces externas}
   \label{tab8} 
   \end{longtable} 
   
   Ahora podemos calcular los puntos de función sin ajustar, TUFP.
   
   \begin{longtable}{|p{3cm}|p{1.5cm}|p{1.9cm}|p{1.5cm}|p{1.9cm}|p{1.5cm}|p{1.9cm}|}
   \hline 
     & \multicolumn{2}{c|}{Bajo} & \multicolumn{2}{c|}{Medio} & \multicolumn{2}{c|}{Alto} \\ 
   \hline 
     & Número & Puntuación & Número & Puntuación & Número & Puntuación \\ 
   \hline 
   Ficheros Lógicos Internos a la Aplicación & 3 & 7 & 0 & 10 & 0 & 15 \\ 
   \hline 
   Ficheros Lógicos Externos a la Aplicación
    & 1 & 5 & 0 & 7 & 0 & 10 \\ 
   \hline 
   Input Externos & 4 & 3 & 0 & 4 & 0 & 6 \\ 
   \hline 
   Outputs Externos
    & 3 & 4 & 0 & 5 & 0 & 7 \\ 
   \hline 
   Queries & 7 & 3 & 0 & 4 & 0 & 6 \\ 
   \hline 
     &   & 71 &   & 0 &   & 0 \\ 
   \hline 
   \caption{Tabla de TUFP}
   \label{tab9} 
   \end{longtable} 
   
   El total de TUFP = 53 , cálculo y tabla generada a partir de la planilla de cálculo.
   
   Grados de influencia y otros cálculos
   
   Calculamos los grados de influencia a partir de la planilla de calculo y podemos obtener lo siguiente:
   
   \begin{longtable}{|c|c|}
   \hline 
   TOTAL GRADOS DE INFLUENCIA & 11 \\ 
   \hline 
   Puntos de Función Ajustados & 53 \\ 
   \hline 
   VAF & 0,76 \\ 
   \hline 
   \caption{Tabla de Grados de influencia y otros cálculos}
   \label{tab10}
   \end{longtable} 
   
\subsubsection{C.T Jones / COCOMO II}

\begin{longtable}{|c|c|}
\hline 
FP & 53 \\ 
\hline 
Loc en POO  & 20 \\ 
\hline 
\caption{Tabla de FP y LOC}
\label{tab11}
\end{longtable} 

\begin{longtable}{|p{8cm}|c|}
\hline 
C.T Jones &   \\ 
\hline 
Métrica  & Cálculo \\ 
\hline 
Estimación de meses de desarrollo & 4,894522709 \\ 
\hline 
Número de personas necesarias para el desarrollo (NP) & 0,3533333333 \\ 
\hline 
Estimación del esfuerzo en personas/mes  & 1,729398024 \\ 
\hline 
Estimación de esfuerzo en horas hombre (160 horas x mes) 160 horas, ya que
, se considera 1 jornada diaria de 8 horas & 276,7036838 \\ 
\hline 
\caption{Tabla de C.T Jones}
\label{tab12}
\end{longtable} 

Estimación de esfuerzo en base a C.T. Jones 1.996 Software Estimating Rules of Thumb” y B. Boehm COCOMO II en fase de Early Design

\begin{longtable}{|c|c|}
\hline 
COCOMO II &   \\ 
\hline 
Métrica  & Cálculo \\ 
\hline 
Conversión de FP a LOC. & 1060 \\ 
\hline 
\caption{Tabla de Metrica y Calculo COCOMO II}
\label{tab13}
\end{longtable}

Calculo de esfuerzo con COCOMO II con entrada de 1260 LOC y con factores de escala nominales.

\begin{longtable}{|p{3cm}|c|c|c|}
 \hline 
 Estimación & Optimista & Conservador & Pesimista \\ 
 \hline 
 Meses de desarrollo & 4,6 & 5,2 & 5,9 \\ 
 \hline 
 Personas necesarias & 0,4 & 0,6 & 0,8 \\ 
 \hline 
 Personas/mes & 2 & 3 & 4,5 \\ 
 \hline 
 Esfuerzo en horas hombre (160 horas x mes) & 294,4 & 499,2 & 755,2 \\ 
 \hline 
 \caption{Tabla de Estimación de COCOMO II}
\label{tab14}
 \end{longtable}
 
\begin{longtable}{|c|c|}
  \hline 
  LOC Equivalentes & 1060 \\ 
  \hline 
 \caption{Tabla de LOC Equivalentes}
\label{tab15}
\end{longtable}

Estimación del costo monetario en base al esfuerzo en HH conservador. Definiremos 2 perfiles:

Programador Junior: Perfil requerido para realizar la codificación crítica en base al diseño del software, gestión del proyecto, requerimientos, el plan del proyecto y los diseños base de la aplicación.

Artista Junior: Perfil requerido para la creación de assets 3d y UI para los diseños creados para el proyecto.

Estimación del coste de producir el software en 5.2 meses con 499 HH, bajo el modelo conservador de COCOMO II 

\subsection{Asignación de recursos}

\begin{longtable}{|p{4cm}|p{2.2cm}|p{2.4cm}|p{2.7cm}|p{2cm}|}
\hline 
Costo del producto  &   &   &   &   \\ 
\hline 
Etapa    & Distribución en \% & Distribución de T (HH) & Costo x Recurso & Recurso \\ 
\hline 
Gestión del proyecto & 3 & 14,97 & Programador Junior & \$44.910 \\ 
\hline 
Requerimientos & 8 & 39,92 & Programador Junior & \$119.760 \\ 
\hline 
Plan del proyecto & 1 & 4,99 & Programador Junior & \$14.970 \\ 
\hline 
Diseño inicial & 8 & 39,92 & Programador Junior & \$119.760 \\ 
\hline 
Diseño detallado & 8 & 39,92 & Programador Junior & \$119.760 \\ 
\hline 
Codificación & 48 & 239,52 & Programador Junior & \$718.560 \\ 
\hline 
Documentación de usuario & 1 & 4,99 & Programador Junior & \$14.970 \\ 
\hline 
Prueba unitaria & 3 & 14,97 & Programador Junior & \$44.910 \\ 
\hline 
Prueba funcional & 3 & 14,97 & Programador Junior & \$44.910 \\ 
\hline 
Prueba de integración & 4 & 19,96 & Programador Junior & \$59.880 \\ 
\hline 
Prueba de la app & 6 & 29,94 & Programador Junior & \$89.820 \\ 
\hline 
Creación de Assets 3D & 4 & 19,96 & Artista Junior & \$39.920 \\ 
\hline 
Creación de UI & 3 & 14,97 & Artista Junior & \$29.940 \\ 
\hline 
  & 100 & 499 &   & \$1.392.210 \\ 
\hline 
\caption{Tabla de Asignacion de Recursos}
\label{tab16}
\end{longtable} 



\subsection{Planificación temporal de actividades}

En la siguiente tabla se especifica lo necesario para prototipar la aplicación.

\begin{longtable}{| p{.20\textwidth}| p{.12\textwidth}| p{.12\textwidth}| p{.39\textwidth}|}
\hline 
	Carta Gantt & 
	Inicio & 
	Termino	& 
	Descripción
\\ 
\hline 
	1.- Integración de piezas (modelos 3d) &
	01/06/2020 &
	20/07/2020 & 
	La app consta con los modelos 3d finales.
\\ 
\hline 
	1.1.- Escanear tarjetas con AR &
 	01/06/2020 & 
 	14/06/2020 & 
	La app tendrá en completo funcionamiento la capacidad de escanear las tarjetas diseñadas.
\\ 
\hline 
	1.2-  Mostrar habitacion del museo &
	14/06/2020 &
	01/07/2020 & 
	Se mostrará el ambiente de museo dentro de la aplicación.
	\\ 
\hline 
	1.2.1- Creación de tarjetas con código &
	01/06/2020 &
	14/06/2020 & 
	Creacion de imagenes apropiadas para la app.
	\\
\hline 
	1.2.2.- Diseño de imágenes &
	01/06/2020 &
	07/06/2020 &
	Se diseñarán las imágenes de forma práctica para una buena lectura por parte del escáner.
\\
\hline 
	1.2.3.- Test de usabilidad &
	07/06/2020 &
	14/06/2020 & 
	Se verificará que las imágenes diseñadas funcionen correctamente dentro de la app.
\\ 
\hline 
	1.3.- Interacción con las piezas del museo &
	28/06/2020 &
	12/06/2020 & 
	Se implementará una modo “visualización de pieza” donde el usuario será capaz de realizar diferentes interacciones con las piezas del museo.
\\
\hline 
	1.3.1.- Zoom in/out de la pieza &
	28/06/2020 &
	12/07/2020 & 
	Implementar la acción de acercar y alejar la pieza con el fin de que el usuario pueda observar mejor.
\\
\hline 
	1.3.2.- Rotación libre de la pieza &
	28/06/2020 &
	12/07/2020 & 
	Implementar la acción de rotar la pieza con el fin de que el usuario pueda verla completamente sin perder ningún detalle.
	\\
\hline 
	1.3.3.- Visualizar info de las piezas &
	28/06/2020 &
	12/07/2020 &
	Implementar un panel descriptivo que muestre al usuario la información básica de cada pieza, (Nombre, Edad, Fecha descubrimiento, Descubridor/Creador, Descripción breve,Descripción extendida ).
\\
\hline 
	1.3.4.- Acción de dejar de visualizar la pieza &
	28/06/2020 &
	12/07/2020 &
	El usuario tiene que tener la capacidad de volver a la zona principal cuando desee.
\\
\hline 
	1.3.5.- Seleccionar pieza &
	28/06/2020 &
	12/07/2020 &
	El usuario deberá poder acceder a a la visualización de la pieza desde la zona principal cliqueando sobre ella.
\\ 
\hline 
	1.4.- Implementar feedback al descubrir pieza &
	5/07/2020 &
	20/07/2020 & 
	El usuario deberá tener la certeza de que descubrió alguna pieza nueva y que esta se agregó a tu “Menú de descubrimientos”.
\\
\hline 
	2.- Compartir en redes sociales &
	01/06/2020 &
	21/06/2020 &
	En cualquier momento el usuario podrá compartir, la pieza que está visualizando, el descubrimiento de una nueva pieza, el hecho de que está ocupando la app.
\\
\hline 
	2.1.- Definir el mensaje a compartir &
	01/06/2020 & 
	07/06/2020 & 
	Se deben definir las intenciones del mensaje a compartir en las diferentes R.R.S.S.
\\
\hline 
	2.2.- Definir redes sociales &
	01/06/2020 &
	04/06/2020 & 
	Se debe definir a qué redes sociales se quiere que el usuario pueda compartir.
\\
\hline 
	2.3.- Implementar coneccion con RRSS &
	07/06/2020 &
	21/06/2020 &
	Implementar la coneccion con las R.R.S.S.
\\
\hline 
	3.- Sala de trofeos &
	01/06/2020 &
	28/06/2020 &
	El usuario tendrá una “Sala de trofeos” o “Menú de descubrimientos” donde podra ver todas las piezas encontradas hasta el momento, desde esta zona podrá acceder a la visualización e información de las piezas.
\\
\hline 
	3.1.- Diseñar interacciones &
	07/06/2020 &
	21/06/2020 &
	Se debe planear el correcto y más cómodo funcionamiento de la “Sala de trofeos”.
\\
\hline 
	4.- Desplazamiento en el museo &
	07/06/2020 &
	21/06/2020 &
	El usuario debe ser capaz de desplazarse por el museo.
\\
\hline 
	4.1.- Programar desplazamiento en el museo &
	21/06/2020 &
	20/07/2020 &
	La habitación que se está visualizando tendrá flechas que servirán de botones para desplazarse a otras habitaciones.
\\ 
\hline 
	4.2.- Implementar feedback de desplazamiento) &
	01/07/2020 &
	20/07/2020 &
	El usuario debe notar que se desplazó a otra habitación mostrando una animación de desplazamiento y denotando que la habitación en la que esta es diferente.
\\
\hline 
	4.3.- Pruebas de uso &
	01/07/2020 &
	20/07/2020 &
	Se debe asegurar que el desplazamiento por el museo sea cómodo para el usuario.
\\
\hline 
	5.- Flujo de juego &
	01/06/2020 &
	14/06/2020 & 
	...
	\\
\hline 
	5.1.- Diseño de flujo de juego &
	01/06/2020 &
	14/06/2020 &
	Se debe diseñar un “Tutorial” para que el usuario entienda las acciones principales de la aplicación y su usabilidad.
\\
\hline 
	5.2 Pruebas de uso &
	07/06/2020 &
	14/06/2020 & 
	Se debe verificar que el usuario entiende la app y que no deja de lado ciertas características.
\\
\hline 
	6.- Flujo de menús &
	01/06/2020 &
	14/06/2020 &
	...
\\ 
\hline 
	6.1.- Diseñar menús &
	01/06/2020 &
	14/06/2020 &
	Se debe diseñar los diferentes menús dentro de la app.
\\
\hline 
	6.2.- Implementar menús &
	14/06/2020 &
	28/06/2020 &
	...
\\ 
\hline 
	6.3.- Pruebas de uso &
	07/06/2020 &
	14/06/2020 & 
	Se debe verificar que el usuario entiende el funcionamiento de los diferentes menús dentro de la app.
\\
\hline 
	7.- Sistema de guardado &
	28/06/2020 &
	20/07/2020 & 
	La aplicación debe poder guardar las piezas que se han encontrado además de la información básica del usuario.
\\ 
\hline 
	7.1.- Programar sistema de guardado &
	28/06/2020 &
	20/07/2020 & 
	...
\\ 
\hline 
	7.2.- Pruebas de uso &
	12/07/2020 &
	20/07/2020 &
	Se debe asegurar que el los datos del usuario serán persistente, estas pruebas se harán para evitar que los datos se pierdan en situaciones inusuales como apagado del dispositivo cierre inesperado y otros.
\\
\hline
\caption{Tabla de Planificación temporal de actividades}
\label{tab17}
\end{longtable} 

