\subsubsection{situacion de pompetitividad entre jovenes}

{\textbf {Resumen:}}
Dos estudiantes están compartiendo y mostrando el uno al otro las piezas que han encontrado en sus museos virtuales, cuando alguno de ellos tiene una pieza que el otro no este le explica en qué parte del museo virtual está y generan una conversación en base a la pieza historica.

{\textbf {Actores:}}
Estudiante-1, Eestudiante-2.

{\textbf {Propósito:}}
Mostrar el uso social de la aplicación dentro del colegio y usar la competitividad como herramienta de aprendizaje.

{\textbf {Referencias cruzadas:}}
R1.1,R1.2, R6.1, R6.2, R6.3, R6.4, R6.6

\paragraph{Caso de Uso Esencial}

\begin{longtable}{|p{5cm}|p{8cm}|}
\hline 
Acción actores & Respuesta del sistema \\ 
\hline 
Los estudiantes navegan por los menús de piezas dentro de la aplicación. & La aplicación va mostrando un listado de piezas marcando las que tienen descubiertas.
estas piezas están separadas por temática y numeradas.
 \\ 
\hline 
Los estudiantes comparan cada casilla, entre ambos dispositivos. & --- \\
\hline 
Al encontrar casillas vacías en los dispositivos del otro le comenta cual es y cómo encontrarla.  & --- \\
\hline 
\end{longtable}

\paragraph{Diagrama de Caso de Uso}

\begin{figure}[H]
\centerline{\includegraphics[width=15cm]{imgs/CasoUso_3.PNG}}
\caption{Diagrama Caso 3}
\label{fig_3_1}
\end{figure}

\paragraph{Modelo Conceptual}

\begin{figure}[H]
\centerline{\includegraphics[width=15cm]{imgs/ModeloConceptualCaso_3_3.png}}
\caption{Modelo Conceptual Caso 3}
\label{fig_3_2}
\end{figure}


\paragraph{Diagrama de Secuencia o Colaboración}

\begin{figure}[H]
\centerline{\includegraphics[width=15cm]{imgs/CasoUso_3_2.PNG}}
\caption{Diagrama de Secuencia Caso 3}
\label{fig_3_3}
\end{figure}

\paragraph{Priorización}
{\textbf {Tipo:}}
Relevante.