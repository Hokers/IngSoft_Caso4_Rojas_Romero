\subsubsection{Excursión de grupo escolar}

{\textbf {Resumen:}}
Un profesor decide llevar a su grupo de estudiantes a una excursión para el museo, algunos estudiantes no se ven entusiasmado con la idea así que el profesor los desafía a encontrar las piezas del museo en la app “Museo en casa”. Durante el trayecto de la excursión los alumnos aprovechan de revisar la información del museo y de las piezas en el para que al llegar al museo ya sepan que es lo que tienen que buscar.

{\textbf {Actores:}}
Estudiantes (plural), Profesor.

{\textbf {Propósito:}}
Apoyo didáctico y lúdico para actividades que son llevadas por instituciones educativas durante el periodo normal de clases.

{\textbf {Referencias cruzadas:}}
R1.1, R1.2, R4.4, R5.2, R5.3, R5.4, R5.5, R5.7, R5.1.2

\paragraph{Caso de Uso Esencial}

\begin{longtable}{|p{5cm}|p{8cm}|}
\hline 
Acción actores & Respuesta del sistema \\ 
\hline 
XXXX & XXXX \\ 
\hline 
\end{longtable}

\paragraph{Diagrama de Caso de Uso}

\begin{figure}[H]
\centerline{\includegraphics[width=15cm]{imgs/CasoUso_5.PNG}}
\caption{Caso-1}
\label{fig}
\end{figure}

\paragraph{Modelo Conceptual}

%\begin{figure}[H]
%\centerline{\includegraphics[width=15cm]{imgs/CasoUso_1_3.PNG}}
%\caption{Caso-1}
%\label{fig}
%\end{figure}

\paragraph{Diagrama de Secuencia o Colaboración}

\begin{figure}[H]
\centerline{\includegraphics[width=15cm]{imgs/CasoUso_5_2.PNG}}
\caption{Caso-1}
\label{fig}
\end{figure}

\paragraph{Priorización}
{\textbf {Tipo:}}
Principal.