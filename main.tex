\documentclass[12pt]{article}
%\usepackage[english]{babel}
\RequirePackage[spanish]{babel}
\usepackage[spanish]{babel}
\usepackage{graphicx}

%\usepackage[pdftex,bookmarks,colorlinks,breaklinks]{hyperref}  % PDF hyperlinks, with coloured links
%\usepackage[spanish]{babel}
\usepackage[pdftex,bookmarks,breaklinks]{hyperref}  % PDF hyperlinks, with coloured links
%\usepackage[spanish]{babel}
\usepackage[utf8]{inputenc}

\usepackage{epstopdf}
\usepackage{fullpage}
\usepackage{url}
\usepackage{colortbl}
\usepackage{xcolor}
\usepackage{lscape}
\usepackage{float}
\usepackage{longtable}
\usepackage{glossaries}

\renewcommand{\baselinestretch}{1.5}
\usepackage{multirow}
\parskip 3ex % espacio entre parrafos.
\makeatletter
\renewcommand\paragraph{\@startsection{paragraph}{4}{\z@}%
	{-2.5ex\@plus -1ex \@minus -.25ex}%
	{1.25ex \@plus .25ex}%
	{\normalfont\normalsize\bfseries}}
\makeatother
\setcounter{secnumdepth}{4} % how many sectioning levels to assign numbers to
\setcounter{tocdepth}{4}    % how many sectioning levels to show in ToC

\begin{document}
%%%%%%%%%%%%%%% PORTADA %%%%%%%%%%%%%%%%%%
\pagestyle{empty}
\begin{figure}
   \centering
   \includegraphics[scale=.5]{imgs/logo_utal.png}
\end{figure}

\begin{center}
Facultad de Ingeniería\\
Escuela de Ingeniería en Bioinformática\\
Ingeniería de Software\\
\bigskip\bigskip\bigskip\bigskip


\rule{14cm}{0.5mm}

\begin{Huge}\textbf{App ``Museo en casa", documento técnico}\end{Huge}

\rule{14cm}{0.5mm}

\bigskip\bigskip\bigskip\bigskip
\bigskip\bigskip\bigskip\bigskip
\bigskip\bigskip\bigskip\bigskip
%\bigskip\bigskip\bigskip\bigskip
%\bigskip\bigskip\bigskip\bigskip

\begin{tabular*}{14cm}{l@{\extracolsep{\fill}}r}
\textbf{\emph{Integrantes:}} & \textbf{\emph{Profesor:}}\\
José Rojas & Felipe Besoain\\
Nicolás Romero & \textbf{\emph{Ayudante:}}\\
                   & José Riffo\\
\end{tabular*}
\end{center}

%%%%%%%%%%%%%%%%%%%%%%%%%%%%%%%%%%%%%%%
\newpage
\pagestyle{plain}
\tableofcontents

%%%%%%%%%%%%%%%%%%%%%%%%%%%%%%%%%%%%%%%
\newpage
\listoffigures 

%%%%%%%%%%%%%%%%%%%%%%%%%%%%%%%%%%%%%%%
\newpage
\listoftables


%%%%%%% INTRODUCCIÓN %%%%%%%%%%%%%%%%
\newpage
\section{Introducción}
\color{red}
La vida actual en la que vivimos está rodeada de aparatos tecnológicos y diversos elementos que han extendido la forma en que las personas interactúan entre ellas y con su entorno, dentro de estas interacciones es sencillo notar que existe un incremento en en interés de las personas en visitar lugares culturales como pueden centros históricos, lugares patrimoniales y museos (datos obtenidos de : SNM subdirección Nacional de Museos).

2019 - 1.210.075\\
2018 - 1.390.952 - 18.73 m\\
2017 - 1.252.418 - 18.47 m\\
2016 - 1.276.068 - 18.21 m\\
2015 - 1.112.809 - 17.97 m\\
2014 - 714.137 - 17.76 m\\
2013 - 719.675 - 17.57 m\\
2012 - 685.103 - 17.40 m\\
2011 - 632.765 - 17.23 m\\
2010 - 561.461 - 17.06 m\\
2009 - 682.144 - 16.89 m\\
2008 - 667.679 - 16.71 m\\
https://www.museoschile.gob.cl/sitio/Contenido/Institucional/90496:Estadisticas-generales\\
\color{black}

\subsection{Propósito}
\color{red}
El propocito de esta aplicacion es conectar a las personas por esta nueva via de informacion acercando los museos locales hasta sus propias manos.
\color{black}

\subsection{Descripción breve del problema}
\color{red}
(posible problema) el COVID-19 a traido concigo grandes cambios con respecto a como es nuestro estilo de vida, y generando para la mayoria de la poblacion mundial el tener que adapatarse y usar aun mas de lo que ya se usaban la tecnologia para poder conectarse con el respo de personas,  junto con esta adaptacion tambien debemos destacar que cosas como salir de la casa para ir al coleguio o salir de la casa a en un paseo familiar se transformaron en actividades que no estan permitidas. (datos del gob de chile ->) podemos notar con los datos del gob la gran baja en la asistencia a los museos desde marzo del 2020 en chile (incertar cifra), este fenomeno a alejado a los museos de las personas que podrian estar interesadas.
\color{black}

%%%%%%% PLANIFICACIÓN DE TRABAJO %%%%%%%%%%%%
\newpage
\section{Planificación del Trabajo}

\subsection{Descripción del grupo de trabajo}
El grupo para este proyecto esta conformado por:

José Rojas : Programmer, Artist, Game Designer.

Nicolás Romero : Programmer, Artist, Game Designer.

\subsection{Estimación de esfuerzo}

\subsection{Asignación de recursos}

\subsection{Planificación temporal de actividades}

%%%%%%%%%% ANALISIS %%%%%%%%%%%%%%
\newpage
\section{Análisis}

\subsection{Contexto}
\subsubsection{Descripción General}
\subsubsection{Descripción de Clientes y Usuarios:}

\subsection{Especificación de Requerimientos}
\subsubsection{Funciones del Sistema}

\begin{longtable}{|c|p{10cm}|c|}
\hline 
Ref\# & Función & Categoría (E/O/S) \\ 
\hline 
1.1 & Iniciar juego al presionar en el medio de la pantalla con la app abierta. & E \\ 
\hline 
1.2 & Mostrar título del juego & E \\ 
\hline 
1.3 & Escanear código & E \\ 
\hline 
1.4 & Mostrar museo correspondiente al código Escaneado & E \\ 
\hline 
2.1 & Visualizar pieza en 3d situada en el museo & E \\ 
\hline 
2.2 & Obtener modelo de las piezas & O \\ 
\hline 
2.3 & Interactuar con la pieza & E \\ 
\hline 
2.4 & Visualizar pieza en 3d en el panel de información & E \\ 
\hline 
2.5 & Desplegar visualizador de pieza & E \\ 
\hline 
2.6 & Desplegar ventana de manipulación de pieza & E \\ 
\hline 
2.7 & Obtener información de las piezas & O \\ 
\hline 
2.8 & Mostrar información de las piezas & E \\ 
\hline 
3.1 & Activar/Desactivar menú desplegable de opciones y características & E \\ 
\hline 
3.2 & Silenciar aplicación & S \\ 
\hline 
3.3 & Redireccionar a pagina de la aplicación & S \\ 
\hline 
3.4 & Mostrar Guia/Tutorial de uso básico de la app & E \\ 
\hline 
3.5 & Tutorial de manejo de pieza & E \\ 
\hline 
4.1 & Construir mensaje al compartir en RRSS & O \\ 
\hline 
4.2 & Obtener información de la pieza para compartir en RRSS & O \\ 
\hline 
4.3 & 
Desplegar menú para compartir en RRSS
 & E \\ 
\hline 
4.4 & Obtener información del museo para compartir en RRSS & O \\ 
\hline 
5.1 & Desplegar menú de museos & E \\ 
\hline 
5.2 & Mostrar museos visitados y no visitados & E \\ 
\hline 
5.3 & Obtener información de museos & O \\ 
\hline 
5.4 & Desplazarse entre los museos & E \\ 
\hline 
5.5 & Cerrar ventana de museos & E \\ 
\hline 
5.6 & Descargar QR del museo & O \\ 
\hline 
5.7 & Buscador de museo & E \\ 
\hline 
5.8 & Zoom Museo & E \\ 
\hline 
5.9 & Desplazarse por el museo virtual & E \\ 
\hline 
5.1.1 & Mostrar la descarga del QR del museo & E \\ 
\hline 
5.1.2 & Mostrar información de los museos & E \\ 
\hline 
6.1 & Desplegar menu de piezas & E \\ 
\hline 
6.2 & Retroceder al menú de museos & E \\ 
\hline 
6.3 & Desplazarse entre las piezas & E \\ 
\hline 
6.4 & Mostrar piezas obtenidas y no descubiertas & E \\ 
\hline 
6.5 & Obtener información de piezas & O \\ 
\hline 
6.6 & Cerrar ventana de piezas & E \\ 
\hline 
6.7 & Zoom Pieza & E \\ 
\hline 
6.8 & Rotación pieza & E \\ 
\hline 
7.1 & Desplegar menu para visualizar logros & E \\ 
\hline 
7.2 & Cerrar ventana de logros & E \\ 
\hline 
7.3 & Obtener información de logros & O \\ 
\hline 
7.4 & Mostrar logros obtenidos y no completados & E \\ 
\hline 
7.5 & Desplazarse entre los logros & E \\ 
\hline 
7.6 & Visualizar nuevo logro & E \\ 
\hline 
\end{longtable} 

\subsubsection{Atributos del Sistema}

\subsubsection{Atributos por Función}

\newpage
\subsection{Actores}

\subsection{Casos de Uso}
\subsubsection{Caso de Uso Esencial}
\subsubsection{Diagrama de Caso de Uso}
\subsubsection{Contrato}
\subsubsection{Modelo Conceptual}
\subsubsection{Diagrama de Secuencia o Colaboración}
\subsubsection{Priorización}

\subsection{Modelo de Dominio}
\subsubsection{Entidades Reconocidas}
\subsubsection{Modelo de Dominio}
\subsubsection{Matriz de Rastreabilidad}

%%%%%%%%% VALIDACION %%%%%%%%%%%
\newpage
\section{Validación}

En la presente sección se describirán imágenes del prototipo preparado de la aplicación, estas imágenes se han tomado desde el motor gráfico del juego en su modo de edición. 

\subsection{Prototipo de validación funcional}

\subsubsection{Repositorio}

\url{https://github.com/Hokers/IngSoft_Caso4_Rojas_Romero.git}

\subsubsection{Imagenes}

\begin{figure}[H]
\centerline{\includegraphics[width=15cm]{imgs/Game_2.PNG}}
\caption{Menu principal aplicación.}
\label{game_2}
\end{figure}

\begin{figure}[H]
\centerline{\includegraphics[width=15cm]{imgs/Game_1.PNG}}
\caption{Menu principal aplicación con menu de opciones desplegado.}
\label{game_1}
\end{figure}

\begin{figure}[H]
\centerline{\includegraphics[width=15cm]{imgs/Game_4.PNG}}
\caption{Escena de juego mostrando una habitación del museo.}
\label{game_4}
\end{figure}

\begin{figure}[H]
\centerline{\includegraphics[width=15cm]{imgs/Game_5.PNG}}
\caption{Acercamiento a pieza a descubrir del museo.}
\label{game_5}
\end{figure}

\begin{figure}[H]
\centerline{\includegraphics[width=15cm]{imgs/Game_3.PNG}}
\caption{Escena de juego mostrando pantalla de tutorial.}
\label{game_3}
\end{figure}

%%%%%%%%%%% DISEÑO %%%%%%%%%%%%%
%%%% Latex no permite el uso de 3 subsection por lo que \paragraph fue modificado para que cumpla con las caracteristicas de una subsection

\newpage
\section{Diseño}

\subsection{Derivación del Modelo de Software}
\subsubsection{Modelo de software inicial}
La aplicación “Museo en casa” a sido diseñada como una app monolítica, es decir que no requiere de una conexión externa para su funcionamiento, a pesar de esto se puede dividir a presentación de 3 capas debido a su funcionamiento interno con características similares.

\begin{figure}[H]
\centerline{\includegraphics[width=15cm]{imgs/Modelo3Capas.png}}
\caption{Modelo de 3 Capas}
\label{fig_3Capas}
\end{figure}

Capa presentación:

GameInfoView: Este visualizador permite mostrar el menú de visualización y es el contenedor donde la información es mostrada al usuario, este se utiliza para mostrar información de las piezas, museos y logros correspondientes.

TutorialView: Este módulo se encarga de mostrar toda la información correspondiente del tutorial, además de guiar al usuario para poder realizar las primeras acciones en la app.

Museum Piece Handler: Este módulo se encarga de la manipulación completa de la pieza de museo seleccionada, permite el zoom in/out, rotación, movimiento, etc.

Gameplay Manager:  Este manager se encarga de el flujo correcto del juego, las acciones del usuario, la información que será mostrada, el funcionamiento de las piezas dentro del museo, la notificación de los logros obtenidos y el correcto cierre de los menús.

Capa dominio:

Search Manager: Se encarga de buscar las piezas, logros y museos que se encuentran almacenados en la base de datos de la app.

Info Manager: Maneja la información correspondiente y se asegura de que esta sea correcta para  los logros, piezas y museos los cuales serán enviados al Gameplay Manager, para ser mostrados al usuario.

Info User Manager: Maneja la información obtenida en el momento que se registra el usuario y la envía a la base de datos de los usuarios de la app.

Vuforia Connect: El sistema se conecta a la Vuforia QR DB que contiene los códigos QR para la carga de información correspondiente.

Capa base de datos:

Info Base DB: Esta base de datos almacena la información correspondiente a los logros, piezas y museos, estos datos serán tomados por el Info Manager para su uso.

Vuforia QR DB: Esta base de datos almacena los códigos QR de cada museo registrado en la app. 


\subsubsection{Diagramas de Clases}

\begin{figure}[H]
\centerline{\includegraphics[width=15cm]{imgs/uml.png}}
\caption{Diagrama de clase UML}
\label{fig_UML}
\end{figure}

\subsubsection{Diagramas de Interacción}

Los siguientes diagramas muestran objetos del sistema, así como los mensajes que se pasan entre ellos dentro del caso de uso.


\paragraph{Diagrama de Interacción Caso 1.1}

\begin{figure}[H]
\centerline{\includegraphics[width=15cm]{imgs/Interaccion_1.png}}
\caption{Diagrama de Interacción Caso 1.1}
\label{fig_I_1}
\end{figure}

\paragraph{Diagrama de Interacción Caso 1.2}

\begin{figure}[H]
\centerline{\includegraphics[width=15cm]{imgs/Interaccion_2.png}}
\caption{Diagrama de Interacción Caso 1.2}
\label{fig_I_2}
\end{figure}

\paragraph{Diagrama de Interacción Caso 1.3}

\begin{figure}[H]
\centerline{\includegraphics[width=15cm]{imgs/Interaccion_3.png}}
\caption{Diagrama de Interacción Caso 1.3}
\label{fig_I_3}
\end{figure}

\paragraph{Diagrama de Interacción Caso 1.4}

\begin{figure}[H]
\centerline{\includegraphics[width=15cm]{imgs/Interaccion_4.png}}
\caption{Diagrama de Interacción Caso 1.4}
\label{fig_I_4}
\end{figure}

\paragraph{Diagrama de Interacción Caso 1.5}

\begin{figure}[H]
\centerline{\includegraphics[width=15cm]{imgs/Interaccion_5.png}}
\caption{Diagrama de Interacción Caso 1.5}
\label{fig_I_5}
\end{figure}

\paragraph{Diagrama de Interacción Caso 1.6}

\begin{figure}[H]
\centerline{\includegraphics[width=15cm]{imgs/Interaccion_6.png}}
\caption{Diagrama de Interacción Caso 1.6}
\label{fig_I_6}
\end{figure}

\paragraph{Diagrama de Interacción Caso 1.7}

\begin{figure}[H]
\centerline{\includegraphics[width=15cm]{imgs/Interaccion_7.png}}
\caption{Diagrama de Interacción Caso 1.7}
\label{fig_I_7}
\end{figure}

\paragraph{Diagrama de Interacción Caso 1.8}

\begin{figure}[H]
\centerline{\includegraphics[width=15cm]{imgs/Interaccion_8.png}}
\caption{Diagrama de Interacción Caso 1.8}
\label{fig_I_8}
\end{figure}

\subsubsection{Diagramas de Estados}
Los diagramas de estado muestran el conjunto de estados por los cuales pasa un objeto durante su vida en una aplicación en respuesta a eventos. En este caso se muestra los estados que que se establecen los casos de uso en cuestión.

\paragraph{Diagrama de Estado Caso 1.1}

\begin{figure}[H]
\centerline{\includegraphics[width=15cm]{imgs/Estados_1_1.png}}
\caption{Diagrama de Estado Parte 1 Caso 1.1}
\label{fig_E_1}
\end{figure}

\begin{figure}[H]
\centerline{\includegraphics[width=15cm]{imgs/Estados_1_2.png}}
\caption{Diagrama de Estado Parte 2 Caso 1.1}
\label{fig_E_1_1}
\end{figure}

\paragraph{Diagrama de Estado Caso 1.2}

\begin{figure}[H]
\centerline{\includegraphics[width=15cm]{imgs/Estados_2.png}}
\caption{Diagrama de Estado  Caso 1.2}
\label{fig_E_2}
\end{figure}

\paragraph{Diagrama de Estado Caso 1.3}

\begin{figure}[H]
\centerline{\includegraphics[width=15cm]{imgs/Estados_3.png}}
\caption{Diagrama de Estado  Caso 1.3}
\label{fig_E_3}
\end{figure}

\paragraph{Diagrama de Estado Caso 1.4}

\begin{figure}[H]
\centerline{\includegraphics[width=15cm]{imgs/Estados_4.png}}
\caption{Diagrama de Estado  Caso 1.4}
\label{fig_E_4}
\end{figure}

\paragraph{Diagrama de Estado Caso 1.5}
\begin{figure}[H]
\centerline{\includegraphics[width=15cm]{imgs/Estados_5.png}}
\caption{Diagrama de Estado  Caso 1.5}
\label{fig_E_5}
\end{figure}

\paragraph{Diagrama de Estado Caso 1.6}

\begin{figure}[H]
\centerline{\includegraphics[width=15cm]{imgs/Estados_6_1.png}}
\caption{Diagrama de Estado 1 Caso 1.6}
\label{fig_E_6_1}
\end{figure}

\begin{figure}[H]
\centerline{\includegraphics[width=15cm]{imgs/Estados_6_2.png}}
\caption{Diagrama de Estado 2 Caso 1.6}
\label{fig_E_6_2}
\end{figure}

\paragraph{Diagrama de Estado Caso 1.7}

\begin{figure}[H]
\centerline{\includegraphics[width=15cm]{imgs/Estados_7.png}}
\caption{Diagrama de Estado Caso 1.7}
\label{fig_E_7}
\end{figure}

\paragraph{Diagrama de Estado Caso 1.8}

\begin{figure}[H]
\centerline{\includegraphics[width=15cm]{imgs/Estados_8.png}}
\caption{Diagrama de Estado Caso 1.8}
\label{fig_E_8}
\end{figure}

\subsection{Refinamientos}

\subsubsection{Lugar de Refinamiento}

\subparagraph{Estructuración de información}
Existe la posibilidad de tener el sistema de guardado de datos de usuario fuera de la aplicación.

\subparagraph{Especificaciones de clases}
Existen clases como las dedicadas a presentar "panel Info" y a controlar "GameManager" que se encargan de demasiadas acciones dentro de la aplicación.

\subparagraph{Feedback de usuario}
Existen posibles mejoras dentro de la aplicación orientada a la comodidad del usuario en pos de una control de desplazamiento más cómodo.

\subparagraph{Eficiencia técnica}
existen diferentes consideraciones orientada al uso del procesador dentro de lo que a prestar apoyo gráfico se refiere.


\subsubsection{Para cada Lugar}
\paragraph{Refinamientos considerados}

Dentro de las características principales de esta aplicación se han descubierto posibles mejoras a futuro dentro de los diferentes módulos de desarrollo, en el siguiente  texto se describe la posible mejora y se localiza su ubicación en el proyecto.

Los siguientes refinamientos fueron evaluados “a pulso”.

\subparagraph{\textbf{Información de usuario}}
Hasta ahora la información de los usuarios solo es administrada de manera local en el dispositivo que ejecuta la aplicación, esta puede ser lleva a un servidor externo para poder tener un seguimiento directo de los usuarios de la aplicación además de abrir la posibilidad de que los usuarios no pierdan la información si esta es borrada del dispositivo.
estos cambios implica la adaptación de los módulos de conexión a la información para un funcionamiento con estos servidores externos y sigue los lineamiento de un patrón enfocado a la seguridad informática.

\subparagraph{\textbf{Desplazamiento del jugador}}
Dentro de los interacciones básicas de el jugador con el museo existe la posibilidad de desplazarse entre habitaciones, esta interacción puede ser extendida a las herramientas de control táctil “Slide”, el ampliar este input genera una mayor facilidad al usuario para desplazarse dentro de la aplicación.

\subparagraph{\textbf{Visualización de modelos 3D}}
Existen diferentes lugares dentro del flujo de la aplicación que presentan al usuario los modelados digitales a los usuarios, a pesar de esto no en todos estos momentos se interacciona con este elemento por lo que vuelve ese renderizado en un costo de proceso extra para el dispositivo, esto elementos podrán ser reemplazados por imágenes pre renderizadas para reducir el costo de procesamiento de estos elementos.

\subparagraph{\textbf{Especificación de Presentadores}}
La clase “InfoPanel” es la encargada de presentar  al usuario la información necesaria, esta se aplica al momento de presentar la información de los diferentes elementos de la aplicación (trofeos, piezas, museos), esta arquitectura limita las interacciones específicas que pueden estar relacionadas a cada uno de los tipos de elementos (por ejemplo el poder compartir relacionado mayormente a los trofeos y las piezas relacionadas a obtener más información) por lo que se optara a la implementación de diferentes presentadores de información que obedezcan a una misma lógica pero que cada clase se especifique a destacar estas relaciones.

\subparagraph{\textbf{Especificación de controladores}}
La clase “GameplayManager” controla toda las interacciones básicas del usuario dentro del museo lo que genera una responsabilidad extra para esta clase, es posible dividir esta carga creando controladores de movimiento e interacción lo que facilitará la integración de estas funcionalidades al momento de ser codificadas.

\subparagraph{\textbf{Consolidación de clases establecidas en el UML}}
Durante la realización del prototipo se crearon las clases definidas anteriormente en el UML, pero debido a fallas en la comunicación del equipo, se realizo el código que correspondía a la clase `Gameplay Manager' en una nueva clase que no se encuentra en el UML denominada `MuseumRoomSwitcher', el contenido de esta clase realiza lo que se tenia planeado para la clase `Gameplay Manager', luego de que el equipo analizo esta situación, decidió eliminar la clase `MuseumRoomSwitcher' pero antes migrando el código de esta a la clase que estaba estipulada en el diagrama UML, de esta forma se mantiene la planificación realizada por el equipo y se mantienen los estándares del código en cuestión.

%\paragraph{Selección y descripción de una opción}
%******* esto esta descrito arriba asi que no se si separar lo de arriba o ignoralo *******


%%%%%%%%%%%%%% IMPLEMENTACION %%%%%%%%%%
\newpage
\section{Implementación}
\subsection{Código fuente completo (parcial)}
\subsection{Modelo de implementación}

\subsection{Dependencias}

\begin{itemize}

\item Vuforia: Se utilizo esta libreria para implementar las funcionalidades de Realidad Aumentada dentro del motor gráfico Unity.

\item Android SDK: Se utilizo esta SDK para realizar los builds para los dispotivos de Android.

\item Unity Engine package: Se utilizo esta libreria para todas las funcionalidades que eran necesarias para la app, ya sea interacción con la UI, elementos del juego, etc.

\item Native Share for Android & iOS: Se utilizo esta libreria para poder compartir las piezas de museos que hayan sido descubiertas por los usuarios de la app.


\end{itemize}

%%%%%%%%%%%%%%%% ANEXOS %%%%%%%%%%%%%%%%5
\newpage
\section{Anexos}


%\newglossaryentry{gls-EI} {
%  name={Entrada externa},
%  description={Corresponden a los datos entregados a la aplicación.},
%}
%\newacronym[longplural={Entrada externa}]{EI_Label}{EI}{Entrada externa}

%\gls{EI_Label}


\subsection{Glosario}



\begin{enumerate}
	\item \textbf{APP:} Acrónimo de aplicación.
	\item \textbf{AR:} Realidad aumentada.
	\item \textbf{Codificación:} Acción de generar código de programación.
	\item \textbf{DB:} Base de datos.
	\item \textbf{DET:} Data Element Types, Son la cantidad de datos relacionados a un elemento de la aplicación.
	\item \textbf{EI:} Entrada externa, son los datos entregados a la aplicación.
	\item \textbf{EO:} Salida externa, son los datos entregados por la aplicación.
	\item \textbf{EQ:} Consulta externa, hace referencia a las consultas que realice la aplicación a otros sistemas.
	\item \textbf{EIF:} Ficheros de interfaces externas, grupo de datos relacionados lógicamente, se mantienen fuera de la aplicación.
	\item \textbf{FTR:} File Type Referenced, son la cantidad de conecciones a los diferentes grupos de datos.
	\item \textbf{FP:} Function points, es el valor de medida entregado por las funcionalidades de una aplicación, están ayudan a definir la complejidad de un proyecto.
	\item \textbf{HH:} Horas hombre, es una unidad de medida para medir el esfuerzo de un trabajo según las horas de trabajo por persona.
	\item \textbf{ILF:} Ficheros lógicos internos, grupo de datos relacionados lógicamente, se mantienen dentro de la aplicación.
	\item \textbf{LOC:} Lines of code, es una unidad de medida que denota el valor de un código por su cantidad de líneas.
	\item \textbf{Pieza, PH  o Pieza Histórica:} Es el nombre usado para referirse a los elementos del juego que representan a los propios elementos de los museos que están en exposición.
	\item \textbf{QR:} Imagen capaz de ser detectada por un dispositivo electrónico. 
	\item \textbf{RRSS:} Acrónimo de redes sociales.
	\item \textbf{UI:} User Interface, es el tipo de vista que se ocupa en una aplicación que permite al usuario interactuar con la aplicación.
	\item \textbf{UML:} "Unified Modeling Language" Es un estándar que se ha adoptado a nivel internacional por numerosos organismos y empresas para crear esquemas, diagramas y documentación relativa a los desarrollos de software.
	\item \textbf{UTP:} Unidad tecnico pedagogica.
\end{enumerate}
















\end{document}
